
\documentclass[11pt,a4paper]{article}
\usepackage[utf8]{inputenc}
\usepackage[margin=1in]{geometry}
\usepackage{fancyhdr}
\usepackage{booktabs}
\usepackage{longtable}
\usepackage{array}
\usepackage{multirow}
\usepackage{xcolor}
\usepackage{titlesec}
\usepackage{enumitem}
\usepackage{hyperref}

% GTU Colors
\definecolor{gtublue}{RGB}{0,51,102}
\definecolor{gtuorange}{RGB}{255,102,0}

% Header and Footer
\pagestyle{fancy}
\fancyhf{}
\fancyhead[L]{\textcolor{gtublue}{\textbf{Gujarat Technological University}}}
\fancyhead[R]{\textcolor{gtuorange}{\textbf{w.e.f. 2024-25}}}
\fancyfoot[L]{\textcolor{gtublue}{\small DI01016011}}
\fancyfoot[C]{\thepage}
\fancyfoot[R]{\textcolor{gtuorange}{\small Python Programming}}

% Adjust header height
\setlength{\headheight}{15pt}

% Section styling
\titleformat{\section}{\Large\bfseries\color{gtublue}}{}{0em}{}
\titleformat{\subsection}{\large\bfseries\color{gtuorange}}{}{0em}{}

% Custom commands
\newcommand{\courseheader}[2]{
    \begin{center}
        \textcolor{gtublue}{\Huge\textbf{#1}}\\
        \vspace{0.3cm}
        \textcolor{gtuorange}{\Large Course Code: #2}\\
        \vspace{0.5cm}
        \rule{\textwidth}{2pt}
    \end{center}
}

\begin{document}
\courseheader{Python Programming}{DI01016011}


\section{Course Information}

\begin{tabular}{|l|p{10cm}|}
\hline
\textbf{Field} & \textbf{Details} \\
\hline
Program & Diploma in Engineering \\
\hline
Branch & Information Technology \\
\hline
Level & Diploma \\
\hline
Semester & 1 \\
\hline
Academic Year & 2024-2025 \\
\hline
Category & PCC \\
\hline
\end{tabular}


\section{Prerequisites}

Basic computer skills, including the ability to write basic statements and expressions.


\section{Rationale}

Computer programming skills are now becoming part of basic education as these skills are increasing of vital importance for future job and career prospects. The Python programming language is one of the most popular programming languages worldwide. The course emphasizes the use of python programming in multiple domains. Python is a modern language for writing compact codes specifically for programming Server-side web apps, Data Analytics and Machine Learning, an important Artificial Intelligence domain. Furthermore, Python has gained popularity in scientific computing, production tools and game programming. This course focuses on developing python programming to do a variety of programming tasks where the students are encouraged to develop basic applications using different open source tools. At the end of the course, the student will be developing adequate basic programming skills using python language.


\section{Course Outcomes}

After completion of the course, students will be able to:

\begin{longtable}{|p{1cm}|p{11cm}|p{2.5cm}|}
\hline
\textbf{No.} & \textbf{Course Outcomes} & \textbf{RBT Level} \\
\hline
\endhead
CO1 & Prepare flowchart and algorithm for solving computing problems. & A \\
\hline
CO2 & Develop python programs to solve simple problems. & A \\
\hline
CO3 & Apply control structure feature of python for developing programs. & A \\
\hline
CO4 & Develop programs in Python using built-in functions, modules, and library functions. & A \\
\hline
CO5 & Develop python programs applying strings and lists manipulation concepts. & A \\
\hline
\end{longtable}

*RBT: Revised Bloom's Taxonomy


\section{Teaching and Examination Scheme}

\begin{center}
\small
\begin{tabular}{|c|c|c|c||p{1.8cm}|p{1.8cm}|p{1.8cm}|p{1.8cm}|c|}
\hline
\multicolumn{4}{|c|}{\textbf{Teaching Scheme (Hours)}} & \multicolumn{5}{c|}{\textbf{Assessment Pattern and Marks}} \\
\hline
\textbf{L} & \textbf{T} & \textbf{PR} & \textbf{C} & \textbf{\centering Theory ESE (E)} & \textbf{\centering Theory PA (M)} & \textbf{\centering Tutorial/Practical PA (I)} & \textbf{\centering Tutorial/Practical ESE (V)} & \textbf{Total} \\
\hline
3 & 0 & 2 & 4 & 70 & 30 & 20 & 30 & 150 \\
\hline
\end{tabular}
\end{center}


\section{Course Content}

\setlength{\LTpre}{0pt}
\setlength{\LTpost}{0pt}
\setlength{\LTleft}{0pt}
\setlength{\LTright}{\fill}
\begin{longtable}[c]{|p{1cm}|p{10.5cm}|p{1.5cm}|p{1.5cm}|}
\hline
\textbf{Unit No.} & \textbf{Content} & \textbf{No. of Hours} & \textbf{\% of Weightage} \\
\hline
\endfirsthead
\hline
\textbf{Unit No.} & \textbf{Content} & \textbf{No. of Hours} & \textbf{\% of Weightage} \\
\hline
\endhead
\hline
\endfoot
1 & \textbf{Problem Solving using Flowchart and Algorithm} \newline 1.1 Introduction, Steps for problem-solving, Algorithm and its characteristics, Importance of algorithm. \newline 1.2 Symbolic representation of a flowchart, Importance and Limitations of flowchart, Flow of control \newline 1.3 Problem solving using pseudocode & 5 & 11 \\
\hline
2 & \textbf{Basics of Python} \newline 2.1 Introduction to python, Python features, Applications of python programming \newline 2.2 Python installation \newline 2.3 Basic structure of python program, Python Comments, Keywords, identifiers, variables, Data types, and Operators. \newline 2.4 Type Conversion & 10 & 17 \\
\hline
3 & \textbf{Flow of Control} \newline 3.1 Introduction to Flow of Control \newline 3.2 Selection \newline \quad 3.2.1 If statement \newline \quad 3.2.2 Elif statement \newline \quad 3.2.3 Nested if statement \newline \quad 3.2.4 match statement \newline 3.3 Repetition \newline \quad 3.3.1 For loop \newline \quad 3.3.2 While loop \newline \quad 3.3.3 Nested loop \newline 3.4 break, continue, and pass Statements & 10 & 24 \\
\hline
4 & \textbf{Functions} \newline 4.1 Introduction to Functions \newline 4.2 User Defined Functions \newline \quad 4.2.1 Arguments and Parameters \newline 4.3 Scope of a Variable \newline \quad 4.3.1 Global Variable \newline \quad 4.3.2 Local Variable \newline Python Standard Library \newline \quad 4.3.3 Built-in functions \newline \quad 4.3.4 Input or output - input(), print() \newline \quad 4.3.5 Mathematical Functions - abs(), divmod(), max(), min(), pow(), sum() \newline \quad 4.3.6 Module \newline $\circ$ math \newline $\circ$ random \newline $\circ$ statistics & 10 & 24 \\
\hline
5 & \textbf{Strings and Lists} \newline 5.1 Introduction to Strings, String Operations, Traversing a String \newline 5.2 Strings Methods and Built-in Functions \newline 5.3 Introduction to List and its Operations \newline 5.4 List Methods and Built-in Functions \newline \quad 5.4.1 Nested and Copying Lists \newline 5.5 List as Arguments to Function & 10 & 24 \\
\hline
\end{longtable}


\section{Suggested Specification Table with Marks (Theory)}

\begin{center}
\small
\begin{tabular}{|c|c|c|c|c|c|}
\hline
\multicolumn{6}{|c|}{\textbf{Distribution of Theory Marks (in \%)}} \\
\hline
\textbf{R Level} & \textbf{U Level} & \textbf{A Level} & \textbf{N Level} & \textbf{E Level} & \textbf{C Level} \\
\hline
20 & 30 & 50 & -- & -- & -- \\
\hline
\end{tabular}
\end{center}

\textit{Where R: Remember; U: Understanding; A: Application, N: Analyze and E: Evaluate C: Create (as per Revised Bloom's Taxonomy)}


\section{References/Suggested Learning Resources}

\subsection{(a) Books:}
\begin{enumerate}
\item Java: The Complete Reference by Herbert Schildt, McGraw Hill Education, 12th Edition, 2021, ISBN: 9781260440232
\item Programming with Java by E Balagurusamy, McGraw Hill Education, 6th Edition, 2019, ISBN: 9789353161491
\item Core Java: Volume I - Fundamentals by Cay S. Horstmann, Prentice Hall, 11th Edition, 2018, ISBN: 9780135166307
\item Head First Java by Kathy Sierra and Bert Bates, O'Reilly Media, 3rd Edition, 2022, ISBN: 9781491910771
\item Thinking in Java by Bruce Eckel, Prentice Hall, 4th Edition, 2006, ISBN: 9780131872486
\end{enumerate}

\subsection{(b) Open source software and website:}
\begin{itemize}
\item \textbf{Softwares}
\begin{itemize}
\item Java Development Kit – JDK 23
\item IntelliJ IDEA Community Edition
\item Visual Studio Code
\item Eclipse IDE
\end{itemize}
\item \textbf{Official Documentation}
\begin{itemize}
\item Official Java Documentation
\end{itemize}
\item \textbf{Online Tutorials}
\begin{itemize}
\item W3Schools Java Tutorial
\item JavaTpoint - Java Programming Tutorials
\item GeeksforGeeks - Java Programming Language
\item Learn Java Online
\item TutorialsPoint - Java Tutorials and Examples
\end{itemize}
\item \textbf{Video Courses}
\begin{itemize}
\item Introduction to Java Programming (Udacity YouTube Playlist)
\item Programming in Java by Debasis Samanta, IIT Kharagpur (NPTEL YouTube Playlist)
\item Java Tutorials For Beginners In Hindi (CodeWithHarry YouTube Playlist)
\end{itemize}
\item \textbf{Comprehensive Courses}
\begin{itemize}
\item Java Programming Nanodegree by Udacity
\item Core Java Specialization by LearnQuest on Coursera
\item Java Masterclass 2025: 130+ Hours of Expert Lessons by Tim Bulchka on Udemy
\item Introduction to Object-Oriented Programming with Java by Georgia Tech on edX
\end{itemize}
\end{itemize}


\section{Suggested Course Practical List}

\begin{longtable}{|p{0.8cm}|p{10.5cm}|p{1.3cm}|p{1.4cm}|}
\hline
\textbf{Sr. No} & \textbf{Practical Outcomes (PrOs)} & \textbf{Unit No.} & \textbf{Hrs.} \\
\hline
\endhead
1 & Prepare flowchart and algorithm for a given problem.\newline • Find the sum of two given numbers.\newline • Find a maximum out of two given numbers.\newline • Find whether a given number is odd or even.\newline • Find a maximum out of three given numbers. & 1 & 2 \\
\hline
2 & Install \& configure python software and Create a program to print your name, date of birth and mobile number. & 2 & 2 \\
\hline
3 & Develop a program to identify data-types in python. & 2 & 2 \\
\hline
4 & Create programs for mathematical operations and conversions.\newline • 1) Create a program to read three numbers from the user and find the average of the numbers.\newline • 2) Create a program to convert temperature from Fahrenheit to Celsius unit using eq: C=(F-32)/1.8 & 2 & 2 \\
\hline
5 & Create programs for conditional statements and comparison operations.\newline • 1) Create a program to identify whether the scanned number is even or odd and print an appropriate message.\newline • 2) Create a program to find a maximum number among the given three numbers. & 3 & 2 \\
\hline
6 & Develop a program to show whether the entered number is prime or not. & 3 & 2 \\
\hline
7 & Develop a program to print odd and even numbers from 1 to N numbers. (Where N is an integer number entered by the user) & 3 & 2 \\
\hline
8 & Develop a program to demonstrate the use of break, continue and pass statements. & 3 & 2 \\
\hline
9 & Develop user-defined functions for mathematical operations.\newline • 1) Develop a user-defined function to find the factorial of a given number.\newline • 2) Create a user-defined function to print the Fibonacci series of 0 to N numbers. (Where N is an integer number and passed as an argument) & 4 & 2 \\
\hline
10 & Write a program using the function that reverses the entered value. & 4 & 2 \\
\hline
11 & Write a program that determines whether a given number is an Armstrong number or not using a user-defined function. & 4 & 2 \\
\hline
12 & Write programs for string manipulation operations.\newline • 1) Write a program to reverse words in a given sentence.\newline • 2) Write a program to check if a substring is present in a given string.\newline • 3) Write a program to count and display the number of vowels, consonants, uppercase, lowercase characters in a string. & 5 & 2 \\
\hline
13 & Create programs for list operations and analysis.\newline • 1) Create a program to find the sum of all elements in a list using a loop.\newline • 2) Create a program to find the smallest and largest element in a given list. & 5 & 3 \\
\hline
14 & Given a list saved in variable: a = [1, 8, 7, 15, 25, 36, 48, 64, 81, 95]. Write a Python program that takes this list and makes a new list that has only the even elements of this list in it. & 5 & 3 \\
\hline
\end{longtable}


\section{List of Laboratory/Learning Resources Required}

\begin{enumerate}
\item Computer laboratory with networked computers (1:1 ratio)
\item Latest Java Development Kit (JDK 17 or above)
\item Integrated Development Environment (IDE):
\begin{itemize}
\item IntelliJ IDEA Community Edition
\item Eclipse IDE for Java Developers
\item Visual Studio Code with Java extensions
\end{itemize}
\item Internet connectivity for accessing online resources
\item Projector/Smart board for demonstrations
\item Java documentation and reference materials
\item Sample programs and code libraries
\item Testing frameworks (JUnit)
\item Version control system (Git)
\item Database connectivity (MySQL/PostgreSQL for advanced topics)
\end{enumerate}


\section{Suggested Project List}

\begin{enumerate}
\item \textbf{Student Information Management System}
\begin{itemize}
\item Design a console-based application to manage student records
\item Features: Add, update, delete, and search student information
\item Use classes, objects, and file handling
\end{itemize}

\item \textbf{Library Management System}
\begin{itemize}
\item Create a system to manage books, members, and borrowing records
\item Implement inheritance and polymorphism concepts
\item Use collections framework for data storage
\end{itemize}

\item \textbf{Simple Banking System}
\begin{itemize}
\item Develop a basic banking application with account management
\item Features: Create account, deposit, withdraw, check balance
\item Implement exception handling for invalid operations
\end{itemize}

\item \textbf{Employee Payroll System}
\begin{itemize}
\item Build a payroll management system for different employee types
\item Use inheritance for different employee categories
\item Implement interfaces for payroll calculations
\end{itemize}

\item \textbf{Online Quiz Application}
\begin{itemize}
\item Create a multiple-choice quiz system
\item Features: Question management, scoring, time limits
\item Use multithreading for timer functionality
\end{itemize}

\item \textbf{Inventory Management System}
\begin{itemize}
\item Develop an inventory tracking system for products
\item Features: Add/remove products, stock management, reports
\item Use serialization for data persistence
\end{itemize}

\item \textbf{Simple Chat Application}
\begin{itemize}
\item Create a basic client-server chat system
\item Implement socket programming and multithreading
\item Features: Multiple users, message broadcasting
\end{itemize}

\item \textbf{Calculator with GUI}
\begin{itemize}
\item Build a graphical calculator application
\item Use Swing or JavaFX for the user interface
\item Implement all basic arithmetic operations
\end{itemize}
\end{enumerate}


\section{Suggested Activities for Students}

\begin{enumerate}
\item \textbf{Programming Practice}
\begin{itemize}
\item Solve daily programming challenges on platforms like LeetCode, HackerRank, or CodeChef
\item Practice coding problems related to each unit after completion
\item Participate in online coding contests
\end{itemize}

\item \textbf{Code Reading and Analysis}
\begin{itemize}
\item Study and analyze open-source Java projects on GitHub
\item Review and understand well-written Java code examples
\item Document code analysis findings in a learning journal
\end{itemize}

\item \textbf{Group Programming Projects}
\begin{itemize}
\item Form teams of 3-4 students for collaborative programming projects
\item Use version control systems (Git) for team collaboration
\item Practice code review and pair programming techniques
\end{itemize}

\item \textbf{Technical Documentation}
\begin{itemize}
\item Write technical documentation for developed programs
\item Create user manuals and API documentation
\item Maintain programming portfolios and project reports
\end{itemize}

\item \textbf{Peer Learning Activities}
\begin{itemize}
\item Organize peer teaching sessions on complex topics
\item Conduct code walk-throughs and debugging sessions
\item Participate in study groups for exam preparation
\end{itemize}

\item \textbf{Industry Connection}
\begin{itemize}
\item Attend Java user group meetups and technical seminars
\item Follow Java community blogs and forums
\item Participate in hackathons and coding competitions
\end{itemize}

\item \textbf{Practical Applications}
\begin{itemize}
\item Develop mobile applications using Java frameworks
\item Create web applications using Java technologies
\item Explore Java's role in different domains (web, mobile, enterprise)
\end{itemize}

\item \textbf{Certification Preparation}
\begin{itemize}
\item Prepare for Oracle Java certification exams
\item Complete online Java courses with certificates
\item Build a professional portfolio on platforms like LinkedIn
\end{itemize}
\end{enumerate}

\end{document}