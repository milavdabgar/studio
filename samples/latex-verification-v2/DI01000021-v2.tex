
\documentclass[11pt,a4paper]{article}
\usepackage[utf8]{inputenc}
\usepackage[margin=1in]{geometry}
\usepackage{fancyhdr}
\usepackage{booktabs}
\usepackage{longtable}
\usepackage{array}
\usepackage{multirow}
\usepackage{xcolor}
\usepackage{titlesec}
\usepackage{enumitem}
\usepackage{hyperref}

% GTU Colors
\definecolor{gtublue}{RGB}{0,51,102}
\definecolor{gtuorange}{RGB}{255,102,0}

% Header and Footer
\pagestyle{fancy}
\fancyhf{}
\fancyhead[L]{\textcolor{gtublue}{\textbf{Gujarat Technological University}}}
\fancyhead[R]{\textcolor{gtuorange}{\textbf{w.e.f. 2024-25}}}
\fancyfoot[L]{\textcolor{gtublue}{\small DI01000021}}
\fancyfoot[C]{\thepage}
\fancyfoot[R]{\textcolor{gtuorange}{\small Mathematics-I}}

% Adjust header height
\setlength{\headheight}{15pt}

% Section styling
\titleformat{\section}{\Large\bfseries\color{gtublue}}{}{0em}{}
\titleformat{\subsection}{\large\bfseries\color{gtuorange}}{}{0em}{}

% Custom commands
\newcommand{\courseheader}[2]{
    \begin{center}
        \textcolor{gtublue}{\Huge\textbf{#1}}\\
        \vspace{0.3cm}
        \textcolor{gtuorange}{\Large Course Code: #2}\\
        \vspace{0.5cm}
        \rule{\textwidth}{2pt}
    \end{center}
}

\begin{document}
\courseheader{Mathematics-I}{DI01000021}


\section{Course Information}

\begin{tabular}{|l|p{10cm}|}
\hline
\textbf{Field} & \textbf{Details} \\
\hline
Program & Engineering \\
\hline
Branch & All \\
\hline
Level & Diploma \\
\hline
Semester & 1 \\
\hline
Academic Year & 2024 \\
\hline
Category & BSC \\
\hline
\end{tabular}


\section{Prerequisites}

Linear equation in two variables, Factorization, Polynomial, Quadratic Equation, Coordinate Geometry, LCM, GCD, Concept of Set.


\section{Rationale}

This course of Mathematics is being introduced for providing a solid foundation in basic mathematics concepts and operations that are crucial for further education and everyday problem-solving. This course is an attempt to include topics which are directly applicable to various fields of engineering, technology, business and sciences and develop logical reasoning and critical thinking abilities. The course is designed focusing on multidisciplinary and competency development to ensure students can effectively use mathematical methods and principles in their vocational and technical fields. The components of course ensure that it is comprehensive, practical and aligned with both academic and professional requirements.


\section{Course Outcomes}

After completion of the course, students will be able to:

\begin{longtable}{|p{1cm}|p{11cm}|p{2.5cm}|}
\hline
\textbf{No.} & \textbf{Course Outcomes} & \textbf{RBT Level} \\
\hline
\endhead
CO1 & Interpret the function graphically, numerically and analytically. & A \\
\hline
CO2 & Demonstrate the ability to algebraically analyse basic functions used in Trigonometry. & A \\
\hline
CO3 & Demonstrate the ability to crack engineering related problems based on concepts of Vectors. & A \\
\hline
CO4 & Solve basic engineering problems under given conditions of straight lines and circle. & A \\
\hline
CO5 & Demonstrate the ability to analyse and illustrate the Functions using the concept of Limit. & A \\
\hline
\end{longtable}

*RBT: Revised Bloom's Taxonomy


\section{Teaching and Examination Scheme}

\begin{center}
\small
\begin{tabular}{|c|c|c|c||p{1.8cm}|p{1.8cm}|p{1.8cm}|p{1.8cm}|c|}
\hline
\multicolumn{4}{|c|}{\textbf{Teaching Scheme (Hours)}} & \multicolumn{5}{c|}{\textbf{Assessment Pattern and Marks}} \\
\hline
\textbf{L} & \textbf{T} & \textbf{PR} & \textbf{C} & \textbf{\centering Theory ESE (E)} & \textbf{\centering Theory PA (M)} & \textbf{\centering Tutorial/Practical PA (I)} & \textbf{\centering Tutorial/Practical ESE (V)} & \textbf{Total} \\
\hline
3 & 1 & 0 & 4 & 70 & 30 & 0 & 0 & 100 \\
\hline
\end{tabular}
\end{center}


\section{Course Content}

\setlength{\LTpre}{0pt}
\setlength{\LTpost}{0pt}
\setlength{\LTleft}{0pt}
\setlength{\LTright}{\fill}
\begin{longtable}[c]{|p{1cm}|p{10.5cm}|p{1.5cm}|p{1.5cm}|}
\hline
\textbf{Unit No.} & \textbf{Content} & \textbf{No. of Hours} & \textbf{\% of Weightage} \\
\hline
\endfirsthead
\hline
\textbf{Unit No.} & \textbf{Content} & \textbf{No. of Hours} & \textbf{\% of Weightage} \\
\hline
\endhead
\hline
\endfoot
1 & \textbf{Determinant and Function} \newline 1.1 Determinant and its value up to 3rd order (Without properties) \newline 1.2 Function and simple examples \newline 1.3 Logarithm as a function \newline 1.4 Laws of Logarithm and related Simple examples & 9 & 23 \\
\hline
2 & \textbf{Trigonometry} \newline 2.1 Units of Angles (degree and radian) \newline 2.2 Trigonometric Functions \newline 2.3 Allied \& Compound Angles, Multiple-Submultiples angles \newline 2.4 Graph of Sine and Cosine \newline 2.5 Periodic Trigonometric function \newline 2.6 Sum and factor formulae \newline 2.7 Inverse Trigonometric function & 12 & 20 \\
\hline
3 & \textbf{Vectors} \newline 3.1 Introduction to Vectors \newline 3.2 Vector Operations \newline 3.3 Angle between two Vectors \newline 3.4 Applications of Scalar and Vector Product (Work Done and Moment of Force) & 9 & 20 \\
\hline
4 & \textbf{Coordinate Geometry} \newline 4.1 Straight line (Two-point form) and slope of straight line \newline 4.2 Slope point form, Intercept form, General form of line \newline 4.3 Condition of parallel and perpendicular lines \newline 4.4 Equations of Parallel lines and Perpendicular lines to the given lines \newline 4.5 Angle between two lines \newline 4.6 Equation of circle with center and Radius \newline 4.7 General equation of circle \newline 4.8 Tangent and normal to a circle & 8 & 20 \\
\hline
5 & \textbf{Limit} \newline 5.1 Limit of a Function \newline 5.2 Standard formulae of Limit and related simple examples & 7 & 17 \\
\hline
\end{longtable}


\section{Suggested Specification Table with Marks (Theory)}

\begin{center}
\small
\begin{tabular}{|c|c|c|c|c|c|}
\hline
\multicolumn{6}{|c|}{\textbf{Distribution of Theory Marks (in \%)}} \\
\hline
\textbf{R Level} & \textbf{U Level} & \textbf{A Level} & \textbf{N Level} & \textbf{E Level} & \textbf{C Level} \\
\hline
20 & 30 & 50 & -- & -- & -- \\
\hline
\end{tabular}
\end{center}

\textit{Where R: Remember; U: Understanding; A: Application, N: Analyze and E: Evaluate C: Create (as per Revised Bloom's Taxonomy)}


\section{References/Suggested Learning Resources}

\subsection{(a) Books:}
\begin{enumerate}
\item Java: The Complete Reference by Herbert Schildt, McGraw Hill Education, 12th Edition, 2021, ISBN: 9781260440232
\item Programming with Java by E Balagurusamy, McGraw Hill Education, 6th Edition, 2019, ISBN: 9789353161491
\item Core Java: Volume I - Fundamentals by Cay S. Horstmann, Prentice Hall, 11th Edition, 2018, ISBN: 9780135166307
\item Head First Java by Kathy Sierra and Bert Bates, O'Reilly Media, 3rd Edition, 2022, ISBN: 9781491910771
\item Thinking in Java by Bruce Eckel, Prentice Hall, 4th Edition, 2006, ISBN: 9780131872486
\end{enumerate}

\subsection{(b) Open source software and website:}
\begin{itemize}
\item \textbf{Softwares}
\begin{itemize}
\item Java Development Kit – JDK 23
\item IntelliJ IDEA Community Edition
\item Visual Studio Code
\item Eclipse IDE
\end{itemize}
\item \textbf{Official Documentation}
\begin{itemize}
\item Official Java Documentation
\end{itemize}
\item \textbf{Online Tutorials}
\begin{itemize}
\item W3Schools Java Tutorial
\item JavaTpoint - Java Programming Tutorials
\item GeeksforGeeks - Java Programming Language
\item Learn Java Online
\item TutorialsPoint - Java Tutorials and Examples
\end{itemize}
\item \textbf{Video Courses}
\begin{itemize}
\item Introduction to Java Programming (Udacity YouTube Playlist)
\item Programming in Java by Debasis Samanta, IIT Kharagpur (NPTEL YouTube Playlist)
\item Java Tutorials For Beginners In Hindi (CodeWithHarry YouTube Playlist)
\end{itemize}
\item \textbf{Comprehensive Courses}
\begin{itemize}
\item Java Programming Nanodegree by Udacity
\item Core Java Specialization by LearnQuest on Coursera
\item Java Masterclass 2025: 130+ Hours of Expert Lessons by Tim Bulchka on Udemy
\item Introduction to Object-Oriented Programming with Java by Georgia Tech on edX
\end{itemize}
\end{itemize}


\section{List of Laboratory/Learning Resources Required}

\begin{enumerate}
\item Computer laboratory with networked computers (1:1 ratio)
\item Latest Java Development Kit (JDK 17 or above)
\item Integrated Development Environment (IDE):
\begin{itemize}
\item IntelliJ IDEA Community Edition
\item Eclipse IDE for Java Developers
\item Visual Studio Code with Java extensions
\end{itemize}
\item Internet connectivity for accessing online resources
\item Projector/Smart board for demonstrations
\item Java documentation and reference materials
\item Sample programs and code libraries
\item Testing frameworks (JUnit)
\item Version control system (Git)
\item Database connectivity (MySQL/PostgreSQL for advanced topics)
\end{enumerate}

\end{document}